\section{Conclusions and Future Work} \label{sec:concl}

In this work, we consider a new application of provenance: to assist in determining the correct sharing mechanisms for data exchanged across organizations and environments. To this end, we introduce the Anonymization Decision Framework (ADF) which is used to manually reason over data flows and data exchange requirements. Through analysis of the ADF, how it is applied, and the information required to make such decisions, we have identified how provenance can be utilized in the future to create more automated decisions.

 In order to express the data situation, the data environments elements: other data, data agents, governance processes, infrastructure have been identified from a real world use case and mapped with W3C PROV elements: entities, bundles, activities, and agents. Further, to fully express the features of data environments for exploiting the ADF based machine enabled reasoning, we observed that the existing PROV constructs are not sufficient and need an extension. 
 
To this end, we analyze how data environments can be represented within the W3C PROV. We identify four different mechanisms within the W3C PROV, and evaluate each with respect to trade-offs of cost, maintenance and suitability for ADF. While two obviously does not pass muster, the other two are viable solutions, with one that utilizes existing W3C PROV structures and require an additional management, and the  second needs an extension.